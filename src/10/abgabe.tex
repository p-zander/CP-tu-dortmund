\documentclass[paper=a4, ngerman]{scrartcl}
\usepackage[utf8]{luainputenc}
\usepackage{amsmath}
\usepackage{graphicx}
\usepackage{float}
\usepackage{placeins}
\usepackage{siunitx}
\sisetup{locale = DE, separate-uncertainty = true, mode = text, quotient-mode   = fraction, range-units = single, range-phrase = {~bis~}, table-auto-round  = true, table-figures-decimal   = 1, per-mode = fraction, output-decimal-marker = {,}, multi-part-units  = brackets, exponent-product  = \cdot}

\renewcommand{\exp}[1]{\ensuremath{\text{e}^{#1}}}
\renewcommand{\theequation}{\arabic{subsection}.\arabic{equation}}
\renewcommand{\thesubsection}{}


\begin{document}

\section{Aufgabe 1}
\subsection{a,b)}
\begin{figure}[htbp]
	\centering
	\includegraphics[width=.75\textwidth]{r_a_1.png}
	\caption{$\langle \vec{r}(t)\rangle / a$}
	\label{fig:label}
\end{figure}
\begin{figure}[htbp]
	\centering
	\includegraphics[width=0.75\textwidth]{r2_a_1.png}
	\caption{$\langle \vec{r}^2(t)\rangle / a^2$}
	\label{fig:label}
\end{figure}
\begin{figure}[htbp]
	\centering
	\includegraphics[width=0.75\textwidth]{r2_fit_1.png}
	\caption{Ausgleichsgerade zur Bestimmung von D}
	\label{fig:label}
\end{figure}
Anhand der Ausgleichsrechnung 
\begin{equation}
	f(x,m,b) = 4mx+b
\end{equation} 
konnte ein Wert von $m \approx \SI{0.179}{}$ ermittelt werden.
zusätzlich ergibt sich:
\begin{align}
D = \frac{ma^2}{\tau}
\end{align}

\FloatBarrier
\subsection{c)}
Für die analytische Lösung der Diffusionsgleichung 
\begin{equation}
\partial_tP=D\nabla^2P
\end{equation}
mit $P(\vec{r},0)=\delta(\vec{r})$ folgt
\begin{equation}
	P(\vec{r},0) = \frac{1}{4\pi D t}exp(\frac{-x^2+y^2}{4Dt})
\end{equation}
\begin{figure}[htbp]
	\centering
	\includegraphics[width=.75\textwidth]{hist10.png}
	\caption{Histogram inklusive Vergleich mit der analytischen Lösung nach 10 Zeitschritten}
\end{figure}
\begin{figure}[htbp]
	\centering
	\includegraphics[width=.75\textwidth]{hist100.png}
	\caption{Histogram inklusive Vergleich mit der analytischen Lösung nach 100 Zeitschritten}
\end{figure}
\begin{figure}[htbp]
	\centering
	\includegraphics[width=.75\textwidth]{hist1000.png}
	\caption{Histogram inklusive Vergleich mit der analytischen Lösung nach 1000 Zeitschritten}
\end{figure}
\FloatBarrier
\section{Aufgabe 2}
\begin{figure}[htbp]
	\centering
	\includegraphics[width=.75\textwidth]{mc_int_pi.pdf}
	\caption{Entwicklung des Fehlers der verschieden Methoden.}
\end{figure}
\begin{figure}[htbp]
	\centering
	\includegraphics[width=.75\textwidth]{mc_int_pi_hist.pdf}
	\caption{Verteilung der Ergebnisse.}
\end{figure}
\begin{figure}[htbp]
	\centering
	\includegraphics[width=.75\textwidth]{mc_int_area_ellipse.pdf}
	\caption{Abhängigkeit des Flächeninhaltes von a und b.}
\end{figure}
\end{document}
