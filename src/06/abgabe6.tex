\documentclass[paper=a4, ngerman]{scrartcl}
\usepackage[utf8]{luainputenc}
\usepackage{amsmath} 
\usepackage{amssymb}

\begin{document}

\section{Analytische Lösung}
Löse 
\begin{align}
\partial _x^2 \phi +\partial \phi_y^2 = 0 
\end{align}
mit Produktansatz:
\begin{align}
\phi (x,y) = \phi(x)\phi(y)
\end{align}
Randbedingungen:
\begin{align}
\phi_x(0) = 0, \phi_x(L) = 0, \phi_y(0) = 0, \phi_y(L) = 1
\end{align}
Nach Randbedingung gilt:
\begin{align}
\phi_x = \sin\left(\sqrt{\lambda} x \right) \hspace{.5cm}\text{mit}\hspace{.5cm} \lambda = \left( \frac{n \pi}{L} \right)^2
\end{align}
\begin{align}
\phi_y = C_1 \sinh\left( \frac{n \pi}{L}y \right) + C_2 \cosh \left( \frac{n \pi}{L}y \right) 
\end{align}
\begin{align}
\phi_y(0) = 0 => C_2 = 0 => 
\end{align}
\begin{align}
\phi_y = C_1 \sinh\left( \frac{n \pi}{L}y \right) 
\end{align}
\begin{align}
\phi_y(1) = C_1 \sinh\left( n \pi \right) = 1
\end{align}
\begin{align}
\phi_n (x,y) = C_n \sinh\left(\frac{n \pi}{L} y \right) \sin\left( \frac{n \pi}{L} x \right) 
\end{align}
\begin{align}
\phi(x,y) = \sum _{n = 1}^{\infty} C_n \sinh\left(\frac{n \pi}{L} y \right) \sin\left( \frac{n \pi}{L} x \right) = 1
\end{align}
\begin{align}
 \sum _{n = 1}^{\infty} \int _0^L C_n \sinh\left(\frac{n \pi}{L} y \right) \sin\left( \frac{n \pi}{L} x \right) \text{sin} \left( \frac{m \pi}{L} x \right) dx = \int _0^L \sin \left( \frac{m \pi}{L} x \right) dx
\end{align}
\begin{align}
\int _0^L C_m \sinh (m \pi) \sin^2 \left( \frac{m \pi}{L} x \right) dx = 
\int_0^L \sin \left( \frac{m \pi}{L} x \right) dx
\end{align}
\begin{align}
\frac{1}{2} C_m \sinh (m \pi) = \frac{1 - \cos(m \pi)}{\pi m} 
\end{align}
\begin{align}
C_m = \frac{2(1-\cos(\pi m))}{\pi m \sinh(m \pi)} 
\end{align}
\begin{align}
\phi_n (x,y) = \sum_{n=1} ^\infty \frac{2(1- \cos(\pi n))}{n \pi \sinh(n \pi)} \sinh(n \pi y) \sin (n \pi x)
\end{align}

\end{document}