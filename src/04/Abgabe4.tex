\documentclass[paper=a4, ngerman]{scrartcl}
\usepackage[utf8]{luainputenc}
\usepackage{amsmath} 
\usepackage{amssymb}
\usepackage{graphicx}
\usepackage{caption}
\captionsetup[figure]{style=default, name=Abb., labelformat=simple, justification=justified, labelsep=colon, position=above, width=\textwidth, labelfont={small}, font={small} }

\usepackage{float}
\usepackage[section]{placeins}
\usepackage{siunitx}
\sisetup{locale = DE, separate-uncertainty = true, mode = text, quotient-mode   = fraction, range-units = single, range-phrase = {~bis~}, table-auto-round  = true, table-figures-decimal   = 1, per-mode = fraction, output-decimal-marker = {,}, multi-part-units  = brackets, exponent-product  = \cdot}

\renewcommand{\exp}[1]{\ensuremath{\text{e}^{#1}}}
\renewcommand{\theequation}{\arabic{subsection}.\arabic{equation}}
\renewcommand{\thesection}{}
\renewcommand{\thesubsection}{}

\begin{document}

\section*{Aufgabe 1}
Analytische Bestimmung der Kraft $\mathrm{F}(\theta)$:
\begin{align}
    \nonumber \mathrm{V}\left(\theta\right)&=\frac12 k\left(\mathrm{d}_1(\theta)-b\right)^2+\frac12 k\left(\mathrm{d}_2(\theta)-b\right)^2\\[5pt]
    \nonumber \mathrm{d}_1(\theta)&=\vert (-\mathrm{a},0) - \mathrm{R}(\cos(\theta),\sin(\theta))\vert = \sqrt{\mathrm{a}^2+\mathrm{R}^2+2\mathrm{aR}\cos(\theta)}\\[5pt]
    \nonumber \mathrm{d}_2(\theta)&=\vert \mathrm{R}(\cos(\theta),\sin(\theta)) - (\mathrm{x},\mathrm{y})\vert =\sqrt{\mathrm{R}^2+\mathrm{x}^2+\mathrm{y}^2-2\mathrm{R}\left(\mathrm{x}\cos(\theta)-\mathrm{y}\sin(\theta)\right)}\\[5pt]
    \nonumber \frac{\partial\mathrm{d}_1}{\partial\theta}&= \frac{-\mathrm{aR}\sin(\theta)}{\mathrm{d}_1}\\[5pt]
    \nonumber \frac{\partial\mathrm{d}_2}{\partial\theta}&= \frac{-\mathrm{R}\left(y\sin(\theta)-x\cos(\theta)\right)}{den}\\[5pt]
    \nonumber \mathrm{F}(\theta)&=\frac{-\partial\mathrm{V}}{\partial\theta}=-k\left(\mathrm{d}_1-b\right)\frac{\partial \mathrm{d}_1}{\partial\theta}-k\left(\mathrm{d}_2-b\right)\frac{\partial \mathrm{d}_2}{\partial\theta}
\end{align}
Zur numerischen Bestimmung von $\mathrm{F}(\theta)$ wurde das 4-Punkt-Verfahren verwendet.

\begin{figure}[htbp]
	\centering
	\includegraphics[trim = 10px 20px 10px 40px, clip, width=.7\textwidth]{1_ab_F.pdf}
	\caption{Verlauf der Kraft.}
	\label{fig:1F}
\end{figure}

\begin{figure}[htbp]
	\centering
	\includegraphics[trim = 10px 20px 10px 40px, clip, width=.7\textwidth]{1_ab_V.pdf}
	\caption{Verlauf der Potentiale.}
	\label{fig:1V}
\end{figure}
In Abbildung \ref{fig:1V} ist zu erkennen, dass für y=0 zwei gleich tiefe stabile Gleichgewichtslagen existieren. Erhöht man nun den Wert für y wird eine stabile Gleichgewichtslage tiefer als die andere [siehe (x,y)=(18,1)]. Wird y allerdings größer als ein kritischer Wert geht das System über zu einer stabilen Gleichgewichtslage [siehe (x,y)=(16,2)]. In Abbildung \ref{fig:1b} ist im Weiteren gut zu erkennen, wie sich das System bei Startwerten von $\theta=2$ und $\theta=4$ verhält. Die ersten beiden Kurven für den Fall zweier Gleichgewichtspositionen zeigen, dass die Systeme sich mit fortschreitendem t in der naheliegenden Ruhelage einpendeln. Wie man sieht entspricht der resultierende Wert von $\theta$ eben denen der Minima in Abbildung \ref{fig:1V}. Ist allerdings nur eine Ruhelage vorhanden schwingen beide Systeme hin zur gleichen Ruhelage.

\begin{figure}[htbp]
	\centering
	\includegraphics[trim = 10px 20px 10px 40px, clip, width=.7\textwidth]{1_b.pdf}
	\caption{Verlauf von $\theta(t)$}
	\label{fig:1b}
\end{figure}

\begin{figure}[htbp]
	\centering
	\includegraphics[trim = 10px 20px 10px 40px, clip, width=.7\textwidth]{1_c.pdf}
	\caption{Verlauf von $\theta(t)$ bei zeitlicher Variation von y. In blau y=[0,2), in rot y=[2,0]. Wie zu erkennen ist wird die ursprüngliche  Gleichgewichtslage nicht wieder erreicht.}
	\label{fig:1c}
\end{figure}


\section*{Aufgabe 2}
\begin{figure}[htbp]
    \centering
    \begin{minipage}{.49\linewidth}
        \centering
        \includegraphics[width=1.0\linewidth]{2_a_0.pdf}
        \caption{}
        \label{fig:}
    \end{minipage}
    \begin{minipage}{.49\linewidth}
        \centering
        \includegraphics[width=1.0\linewidth]{2_a_1.pdf}
        \caption{}
        \label{fig:}
    \end{minipage}
        \begin{minipage}{.49\linewidth}
        \centering
        \includegraphics[width=1.0\linewidth]{2_a_2.pdf}
        \caption{}
        \label{fig:}
    \end{minipage}
    \begin{minipage}{.49\linewidth}
        \centering
        \includegraphics[width=1.0\linewidth]{2_a_3.pdf}
        \caption{}
        \label{fig:}
    \end{minipage}
    \begin{minipage}{.49\linewidth}
        \centering
        \includegraphics[width=1.0\linewidth]{2_a_4.pdf}
        \caption{}
        \label{fig:}
    \end{minipage}
    \begin{minipage}{.49\linewidth}
        \centering
        \includegraphics[width=1.0\linewidth]{2_a_5.pdf}
        \caption{}
        \label{fig:}
    \end{minipage}
    \caption{Phasenraumportraits für angegebene x.}
\end{figure} 

\begin{figure}[htbp]
	\centering
	\includegraphics[width=.8\textwidth]{2_b.pdf}
	\caption{Poincaré-Schnitt}
	\label{fig:poin}
\end{figure}

\begin{figure}[htbp]
	\centering
	\includegraphics[width=.8\textwidth]{2_c.pdf}
	\caption{Bifurkationsplot}
	\label{fig:label}
\end{figure}

\end{document}
