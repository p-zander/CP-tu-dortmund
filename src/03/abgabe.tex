\documentclass[paper=a4, ngerman]{scrartcl}
\usepackage[utf8]{luainputenc}
\usepackage{amsmath} 
\usepackage{amssymb}
\usepackage{graphicx}
\usepackage{caption}
\captionsetup[figure]{style=default, name=Abb., labelformat=simple, justification=justified, labelsep=colon, position=above, width=\textwidth, labelfont={small}, font={small} }

\usepackage{float}
\usepackage[section]{placeins}
\usepackage{siunitx}
\sisetup{locale = DE, separate-uncertainty = true, mode = text, quotient-mode   = fraction, range-units = single, range-phrase = {~bis~}, table-auto-round  = true, table-figures-decimal   = 1, per-mode = fraction, output-decimal-marker = {,}, multi-part-units  = brackets, exponent-product  = \cdot}

\renewcommand{\exp}[1]{\ensuremath{\text{e}^{#1}}}
\renewcommand{\theequation}{\arabic{subsection}.\arabic{equation}}
\renewcommand{\thesection}{}
\renewcommand{\thesubsection}{}

\begin{document}

\section{Aufgabe 2}
\subsection{a+b)}

In Abbildung \ref{fig:2a} sind alle von 0 verschiedenen Komponenten für die beiden Einstellungen zu sehen:
\begin{itemize}
\item $\vec{r} = \vec{e}_x, \vec{v} = \vec{0}$ \hspace{2cm} mit zu großer Schrittweite $h = \num{1.0472}$
\item $\vec{r} = \vec{e}_x, \vec{v} = \frac12\vec{e}_y \Rightarrow \vec{r} \nparallel \vec{v}$ ~\, mit ausreichend kleiner 
Schrittweite $h = \num{0.2618}$
\end{itemize}

Zu sehen ist dass bei zu kleiner Schrittweite ($x_0$) die Amplitude schon nach zwei Oszillationen nicht mehr bei 1 ist. Bei viermal so vielen Schritten ($x_1, y_1$) ist die Genauigkeit aber ausreichend. In Abbildung \ref{fig:2b} ist für diesen Fall die Energieerhaltung zu sehen und in Abbildung \ref{fig:2b_2} die Differenz zu Startenergie.
\vfill
\begin{figure}[htbp]
	\centering
	\includegraphics[width=\textwidth]{2a.pdf}
	\caption{Plot für Aufgabe 2}
	\label{fig:2a}
\end{figure}
\vfill
\newpage

\begin{figure}[htbp]
    \centering
    \begin{minipage}{.49\textwidth}
        \centering
        \includegraphics[width=1.0\textwidth]{2b.pdf}
        \caption{Energien\\~}
        \label{fig:2b}
    \end{minipage}
    \begin{minipage}{.49\textwidth}
        \centering
        \includegraphics[width=\textwidth]{2b_2.pdf}
        \caption{Gesamtenergiedifferenz zu\\$E_{ges}(t=0)$}
        \label{fig:2b_2}
    \end{minipage}
\end{figure} 

\section{Aufgabe 3}
\subsection{a)}

Je kleiner $v$ wird, desto kleiner muss die Schrittweite werden damit die Ellipse wirklich schließt. Zu große Schrittweiten führen dabei zu einer sich vergrößernden Abweichung.

\begin{figure}[!h]
	\centering
	\includegraphics[width=.8\textwidth]{a1.pdf}
	\caption{Ellipse für t=0..37, N=25  und v(0, 1.3, 0)}
	\label{fig:a1}
\end{figure}

\begin{figure}[htbp]
    \centering
    \begin{minipage}{.99\textwidth}
		\centering
		\includegraphics[trim = 10px 10px 10px 30px, clip, width=.8\textwidth]{a2.pdf}
		\caption{Ellipse für t=0...37, N=100 und v=(0, 1.3, 0)}
		\label{fig:a2}
    \end{minipage}\\
    \begin{minipage}{.99\textwidth}
		\centering
		\includegraphics[width=.8\textwidth]{a3.pdf}
		\caption{Ellipse für t=0...37, N=100 und v=(0, 0.88, 0)}
		\label{fig:a3}
    \end{minipage}
\end{figure} 

\FloatBarrier
\subsection{b)}

\begin{figure}[htbp]
	\centering
	\includegraphics[trim = 70px 10px 30px 10px, clip, width=\textwidth]{b.pdf}
	\caption{Plot zu Aufgabenteil b).}
	\label{fig:b}
\end{figure}

\FloatBarrier
\subsection{c)}
In Abbildung \ref{fig:c2} ist die resultierende Ellipse inklusive Lenz-Runge Vektor abgebildet. Es ist zu erkennen, dass der Lenz-Runge Vektor vom rechten Brennpunkt in Richtung des nächsten Bahnpunktes (Perizentrum) zeigt.

\begin{figure}[htbp]
	\centering
	\includegraphics[trim = 10px 10px 10px 30px, clip, width=\textwidth]{c1.pdf}
	\caption{Komponenten des Lenz-Runge Vektors}
	\label{fig:c1}
\end{figure}

\begin{figure}[htbp]
	\centering
	\includegraphics[width=\textwidth]{c2.pdf}
	\caption{Ellipse mit Lenz-Runge Vektor in rot.}
	\label{fig:c2}
\end{figure}

\FloatBarrier
\subsection{d)}
Umkehrung der Geschwindigkeit und erneute Ausführung der Routine liefern folgende Werte:
\begin{align}
	\nonumber \text{Verwendete Parameter:  } \mathrm{t}= 0...100, N=10000\\
    \nonumber \mathrm{v}_\mathrm{u}(tN)=(\SI{-1.02378e-8}{}, -1.3, 0) \\
    \nonumber \mathrm{r}_\mathrm{u}(tN)=(1.0, \SI{-1.3308e-8}{}, 0) 
\end{align}
\begin{figure}[htbp]
	\centering
	\includegraphics[width=\textwidth]{d.pdf}
	\caption{Differenz der Geschwindigkeits- und Ortskomponenten.}
	\label{fig:d}
\end{figure}

\FloatBarrier
\subsection{e)}
Verhalten der Komponenten des Lenz-Runge Vektors und Form der Ellipsen bei einem Potential der Form $\mathrm{V(r)}=-\mathrm{Gm}/\mathrm{r}^{\alpha}$.
Wie in den folgenden Plots zu erkennen ist verschieben sich die Bahnen der Ellipsen, und der Lenz-Runge Vektor ist nicht mehr erhalten.
\begin{figure}[htbp]
	\centering
	\includegraphics[width=\textwidth]{e09LR.pdf}
	\caption{Komponenten des Lenz-Runge Vektors in Abhängigkeit der Zeit t, für $\alpha=0.9$.}
	\label{fig:e09l}
\end{figure}
\begin{figure}[htbp]
	\centering
	\includegraphics[width=\textwidth]{e09EL.pdf}
	\caption{Ellipse für $\alpha=0.9$}
	\label{fig:label1}
\end{figure}
\begin{figure}[htbp]
	\centering
	\includegraphics[width=\textwidth]{e11LR.pdf}
	\caption{Komponenten des Lenz-Runge Vektors in Abhängigkeit der Zeit t, für $\alpha=1.1$.}
	\label{fig:e1LR}
\end{figure}
\begin{figure}[htbp]
	\centering
	\includegraphics[width=\textwidth]{e11EL.pdf}
	\caption{Ellipse für $\alpha=1.1$}
	\label{fig:label2}
\end{figure}

\end{document}
